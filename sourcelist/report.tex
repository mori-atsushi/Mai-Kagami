%---------------------------------------------------------------------
%
%  プロコン ソースコード印刷のサンプル
%
%  Last updated: September 1, 2014
%  Compiled by Shoichi Ito
%
%---------------------------------------------------------------------
\documentclass[a4j]{jsarticle}

\usepackage[dvipdfmx]{proconsrc}
\usepackage{myListings}

%  ソースコードを引用するためのリスト環境の設定
%
%  各自の好みで設定する。たいていはデフォルトのままでOK。
%  設定内容については、listings.sty付属のドキュメントlistings.pdfか以下のURLを参照。
%  http://www.biwako.shiga-u.ac.jp/sensei/kumazawa/tex/listings.html
\lstset{
  language=C,  %  プログラミング言語の設定
  basicstyle=\ttfamily\small,
  keywordstyle=\color{blue}\bfseries,      %  ifやwhileなど言語固有のキーワード
  commentstyle=\color{ForestGreen},    %  コメント文
  identifierstyle=\color{Gray},            %  変数名や関数名など
  stringstyle=\color{BrickRed}\ttfamily,  %  printf()の中身など
  classoffset=1,
  xleftmargin=3.5em,
  frame=none,  % none → tRBL で上下左右の囲み
  framesep=5pt,
  showstringspaces=false,  %  半角スペースをいわゆるゲタで表示したいときにtrue
  numbers=left,
  stepnumber=1,
  breaklines=true,
  numberstyle=\footnotesize,
  tabsize=4,  %  タブサイズ
  lineskip=-0.5zw
}


%%%%%%%%%%%%%%%%%%%%%% ここを書き換える %%%%%%%%%%%%%%%%%%%%%%
%  第何回?
\newcommand{\Kai}{99}

%  部門
\newcommand{\Bumon}{課題}

%  登録番号
\newcommand{\TourokuBangou}{88888}

%  プレゼンテーション順序
\newcommand{\Junban}{77}

%  作品名
%  サブタイトルがない場合は \newcommand{\SubTitle}{} のように空にする。
\newcommand{\MainTitle}{舞鏡}   %  メインタイトル
\newcommand{\SubTitle}{}    %  サブタイトル

%  メンバー5人分
%  5人未満の場合は \newcommand{\MemberE}{} のように空にする。
\newcommand{\MemberA}{森 篤史}
\newcommand{\MemberB}{名前教えて}
\newcommand{\MemberC}{名前教えて}
\newcommand{\MemberD}{名前教えて}
\newcommand{\MemberE}{名前教えて}

%  指導教員
\newcommand{\Teacher}{佐村○○}

%  学校名
\newcommand{\School}{明石工業高等専門学校}

%%%%%%%%%%%%%%%%%%%%%%%%%%%%%%%%%%%%%%%%%%%%%%%%%%%%%%%%%%%%%%

\hypersetup{
pdftitle=\MainTitle \SubTitle,
% pdfsubject=第\Kai 回プログラミングコンテスト \hfill プログラムソースリスト \Bumon 部門 \Junban 番,
pdfsubject=プログラムソースリスト,
pdfauthor=\School
}

\begin{document}

%  表紙
\pagenumbering{roman}  %  ダミー(PDFにしたときにアラビア数字の「1ページ」「2ページ」が重複しないようにするため
% \MakeCoverPage          %  表紙
\pagenumbering{arabic}  %  ページ数カウンタをリセットする

%  目次
\MakeTOC

%%%%%%%%%%%%%%%%%%%%%% ここを書き換える %%%%%%%%%%%%%%%%%%%%%%
%  以下、\InputSource{}の引数としてソースファイル名を並べる。
%  必要に応じて改ページや LaTeX の命令を入れる。
%  たとえば \section{} を使って「クライアント側プログラム」
%  「サーバ側プログラム」のように分ければより読みやすいだろう。

\section{本体}
\subsection{C++}

\lstset{ language=C++ }
\InputSource{Animation.cpp}
\InputSource{Bezier.cpp}
\InputSource{Button.cpp}
\InputSource{CommonText.cpp}
\InputSource{Draw.cpp}
\InputSource{DrawGraph.cpp}
\InputSource{DrawObject.cpp}
\InputSource{DrawText.cpp}
\InputSource{Font.cpp}
\InputSource{Grading.cpp}
\InputSource{Kinect.cpp}
\InputSource{KinectBody.cpp}
\InputSource{MaiKagami.cpp}
\InputSource{Main.cpp}
\InputSource{ModeSelect.cpp}
\InputSource{Nfc.cpp}
\InputSource{PartMain.cpp}
\InputSource{PartOption.cpp}
\InputSource{PartOptionPop.cpp}
\InputSource{PartPause.cpp}
\InputSource{PartPlay.cpp}
\InputSource{PartResult.cpp}
\InputSource{PartResultMain.cpp}
\InputSource{PauseScreen.cpp}
\InputSource{PlayScreen.cpp}
\InputSource{PlayScreenObject.cpp}
\InputSource{Result.cpp}
\InputSource{Scene.cpp}
\InputSource{SettingPop.cpp}
\InputSource{Song.cpp}
\InputSource{Songs.cpp}
\InputSource{SongSelect.cpp}
\InputSource{SongSelectCommon.cpp}
\InputSource{SongSelectCover.cpp}
\InputSource{SongSelectMain.cpp}
\InputSource{StartScreen.cpp}
\InputSource{ThroughDetail.cpp}
\InputSource{ThroughMain.cpp}
\InputSource{ThroughOption.cpp}
\InputSource{ThroughPause.cpp}
\InputSource{ThroughPlay.cpp}
\InputSource{ThroughResult.cpp}
\InputSource{ThroughResultMain.cpp}
\InputSource{ThroughResultObject.cpp}
\InputSource{ThroughStart.cpp}
\InputSource{Top.cpp}
\InputSource{TopMain.cpp}
\InputSource{Touch.cpp}
\InputSource{User.cpp}

\subsection{ヘッダファイル}

\InputSource{Animation.h}
\InputSource{Bezier.h}
\InputSource{Button.h}
\InputSource{CommonText.h}
\InputSource{Draw.h}
\InputSource{DrawGraph.h}
\InputSource{DrawObject.h}
\InputSource{DrawText.h}
\InputSource{Font.h}
\InputSource{Grading.h}
\InputSource{Kinect.h}
\InputSource{KinectBody.h}
\InputSource{MaiKagami.h}
\InputSource{Main.h}
\InputSource{ModeSelect.h}
\InputSource{Nfc.h}
\InputSource{PartDefine.h}
\InputSource{PartMain.h}
\InputSource{PartOption.h}
\InputSource{PartOptionPop.h}
\InputSource{PartPause.h}
\InputSource{PartPlay.h}
\InputSource{PartResult.h}
\InputSource{PartResultDefine.h}
\InputSource{PartResultMain.h}
\InputSource{PauseScreen.h}
\InputSource{PlayScreen.h}
\InputSource{PlayScreenObject.h}
\InputSource{Result.h}
\InputSource{Scene.h}
\InputSource{SeetingPop.h}
\InputSource{Song.h}
\InputSource{Songs.h}
\InputSource{SongSelect.h}
\InputSource{SongSelectCommon.h}
\InputSource{SongSelectCover.h}
\InputSource{SongSelectDefine.h}
\InputSource{SongSelectMain.h}
\InputSource{StartScreen.h}
\InputSource{stdafx.h}
\InputSource{ThroughDefine.h}
\InputSource{ThroughDetail.h}
\InputSource{ThroughMain.h}
\InputSource{ThroughOption.h}
\InputSource{ThroughPause.h}
\InputSource{ThroughPlay.h}
\InputSource{ThroughResult.h}
\InputSource{ThroughResultDefine.h}
\InputSource{ThroughResultMain.h}
\InputSource{ThroughResultObject.h}
\InputSource{ThroughStart.h}
\InputSource{Top.h}
\InputSource{TopMain.h}
\InputSource{Touch.h}
\InputSource{User.h}


\section{Webサーバ}

\subsection{HTML}

\lstset{ language=HTML }
\InputSource{list.html}

\subsection{PHP}

\lstset{ language=PHP }
\InputSource{api_video.php}
\InputSource{api_history.php}
\InputSource{api_add.php}
\InputSource{list.php}
\InputSource{login.php}
\InputSource{logout.php}
\InputSource{main.php}
\InputSource{make_history_table.php}
\InputSource{music.php}
\InputSource{video.php}



\subsection{CSS}
\lstset{ language=CSS }
\InputSource{css/style.css}

\subsection{Javascript}
\lstset{ language=Javascript }
\InputSource{js/drawerNav.js}
\InputSource{js/scroll.js}
% \InputSource{js/lib/jquery.min.js}


\section{NFCリーダ用 Android アプリケーション}

\lstset{ language=Java }
\InputSource{MainActivity.java}
\InputSource{MyNfc.java}
\InputSource{UsbMessage.java}

\lstset{ language=XML }
\InputSource{AndroidManifest.xml}


%%%%%%%%%%%%%%%%%%%%%%%%%%%%%%%%%%%%%%%%%%%%%%%%%%%%%%%%%%%%%%

\end{document}
